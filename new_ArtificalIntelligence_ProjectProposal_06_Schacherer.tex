\documentclass[a4paper,11pt]{scrartcl}
\usepackage[top=2cm,bottom=2cm,left=2.5cm,right=2.5cm]{geometry} % Seitenränder einstellen
\usepackage[english]{babel} % Worttrennung nach der neuen Rechtschreibung und deutsche Bezeichnungen
\usepackage[utf8]{inputenc} % Umlaute im Text
%\usepackage[scaled]{helvet} % Schriftart Helvetica
%\renewcommand*\familydefault{\sfdefault} %% Only if the base font of the document is to be sans serif
\usepackage[T1]{fontenc} % Trennung von Umlauten
\usepackage[dvipsnames]{xcolor} % Farbe in Dokument
\definecolor{tumblau}{rgb}{0,0.40234375,0.7734375} % eigene Farbe definieren
\parindent 0pt % kein Einrücken bei neuem Absatz
\usepackage{amsmath} % zusätzliche mathematische Umgebungen
\usepackage{amssymb} % zusätzliche mathematische Symbole
\usepackage{bbold} % zusätzliche mathematische Symbole
\usepackage{units} % schöne Einheiten und Brüche
\usepackage[square]{natbib} % wissenschaftliches Literaturverzeichnis
%\usepackage[printonlyused]{acronym} % Abkürzungsverzeichnis
\usepackage{icomma} % kein Leerzeichen bei 1,23 in Mathe-Umgebung
\usepackage{wrapfig} % von Schrift umflossene Bilder und Tabellen
\usepackage{picinpar} % Objekt in Fließtext platzieren (ähnlich zu wrapfig)
\usepackage{scrhack} % verbessert andere Pakete, bessere Interaktion mit KOMA-Skript
\usepackage{float} % bessere Anpassung von Fließobjekten
\usepackage{pgf} % Makro zur Erstellung von Graphiken
\usepackage{tikz} % Benutzeroberfläche für pgf
\usepackage[
margin=10pt,
font=small,
labelfont=bf,
labelsep=endash,
format=plain
]{caption} % Einstellungen für Tabellen und Bildunterschriften
\usepackage{subcaption} % Unterschriften für mehrere Bilder
\usepackage{enumitem} % no indentation at description environment
\usepackage[onehalfspacing]{setspace} % Änderung des Zeilenabstandes (hier: 1,5-fach)
\usepackage{booktabs} % Einstellungen für schönere Tabellen
\usepackage{graphicx} % Einfügen von Grafiken -> wird in master-file geladen
\usepackage{url} % URL's (z.B. in Literatur) schöner formatieren
\usepackage[pdftex]{hyperref} % Verweise innerhalb und nach außerhalb des PDF; hyperref immer als letztes Paket einbinden
\hypersetup{
pdftitle = {},
pdfauthor = {},
pdfsubject = {},
pdfproducer = {LaTeX},
pdfkeywords = {},
colorlinks,
linkcolor = black,
citecolor = black,
filecolor = black,
urlcolor = blue
} % Einstellungen Dokumenteigenschaften und Farbe der Verweise
%\usepackage{pythonhighlight} % python highlighting

% define bordermatrix with brackets
\makeatletter
\def\bbordermatrix#1{\begingroup \m@th
  \@tempdima 4.75\p@
  \setbox\z@\vbox{%
    \def\cr{\crcr\noalign{\kern2\p@\global\let\cr\endline}}%
    \ialign{$##$\hfil\kern2\p@\kern\@tempdima&\thinspace\hfil$##$\hfil
      &&\quad\hfil$##$\hfil\crcr
      \omit\strut\hfil\crcr\noalign{\kern-\baselineskip}%
      #1\crcr\omit\strut\cr}}%
  \setbox\tw@\vbox{\unvcopy\z@\global\setbox\@ne\lastbox}%
  \setbox\tw@\hbox{\unhbox\@ne\unskip\global\setbox\@ne\lastbox}%
  \setbox\tw@\hbox{$\kern\wd\@ne\kern-\@tempdima\left[\kern-\wd\@ne
    \global\setbox\@ne\vbox{\box\@ne\kern2\p@}%
    \vcenter{\kern-\ht\@ne\unvbox\z@\kern-\baselineskip}\,\right]$}%
  \null\;\vbox{\kern\ht\@ne\box\tw@}\endgroup}
\makeatother



\title{\vspace{-1cm}Artifical Intelligence}
\subtitle{Exercise 6: Project Proposal} \date{\today}

\begin{document}
\maketitle

\section*{Team}
Sebastian Bek, Marvin Klaus, Daniela Schacherer
\section*{Problem Definition}

Nowadays, we get constantly more used to having AI systems ease our every day lives. In particular in the context of electronic devices we use the available features for facilitated text-processing. This can for instance be observed on mobile phone or search engines. \\
In this context we propose the development of an text autocompletion and prediction system. For that purpose, we will build a model suggesting possible word-endings given a prefix. In advance, it will predict the following word relative to the cursor position.

\section*{Dataset and Agent Environment}

By means of collection of data (sets), we will try to design a text bot gathering multiple english text sources in diverse text files. We will use several datasets originating from e.g.:\\
\url{https://nlp.stanford.edu/links/statnlp.html#Corpora}\\
We will divide these data into a sufficiently large training set and a test set. 

\section*{Approach}

We are planning to adress this task by using an machine learning approach. We will implement a neural networks and train it using the selected set of training data. More specifically, we would like to use a long short-term memory (LSTM) network - a special type of recurrent neural networks - as they are commonly used in language recognition tasks, according to literature. 
Since we don't consider ourselves experts in AI, we can't assess how well-suited our approach is for the given task. However, as we are interested in neural networks, we would like to try this method in order to gain a deeper insight into the topic.\\

\textbf{Possible Challenges:}\\
A possible challenge arises from words, which are occurring much more frequently than others, e.g. "in, the, that", since these do not indicate any hint about possibly following words. We have to cope with this task by finding an appropriate solution. Other than that, we would have to deal with spelling mistakes of the user. We could think of a SpellCheck routine, suggesting the correctly spelled word and additionally the predicted next word.


\section*{Evaluation and expected results}
Firstly we would access the accuracy (amount of reasonable predictions) of our network over time for the training set as well as for the test set. We aim at minimizing the difference between the accuracy for the training set and the accuracy for the test set. We will classify a prediction as reasonable if the input text together with the predicted word appear in the test set.  

\section*{Hardware}

We do not think we will require any specific hardware system, yet our approach's (execution and evaluation) requirements will not be especially demanding.

\end{document}